\documentclass[12pt]{article}
 
\usepackage[utf8]{inputenc}
\usepackage[T1]{fontenc}
\usepackage[frenchb,spanish]{babel}
\usepackage{setspace}
\usepackage{geometry}
\usepackage{listings}
\usepackage{verbatim}
\usepackage{fancyhdr}
\usepackage{graphicx}
\usepackage[squaren,Gray]{SIunits}
\usepackage{amsfonts}
\usepackage{bm}
\usepackage{amsmath}
\usepackage{float}
\usepackage{titling}
\usepackage{color}
\usepackage{fix-cm}
\newcommand{\bigsize}{\fontsize{35pt}{20pt}\selectfont}
\usepackage{multicol}
\setlength{\parindent}{0pt} 
\usepackage{multicol}
\usepackage{verbatimbox,lipsum}


\let\begintitlepage\relax

\geometry{hmargin=2.5cm,vmargin=2.5cm}



\begin{document}

\begin{titlepage}
    \begin{center}
       \vspace*{0.5cm}
       
       \Large
       \rule{\linewidth}{1pt} \bsc{CONCURSO DE MODELIZACIÓN MATEMÁTICA} \\
    
       
       \large
       \vspace*{0.5cm}
       Identificador del grupo : CMM2021 - 693968\\
       \vspace*{0.5cm}
       UCM - Facultad de Ciencias Matemáticas - 2021 \\ 
       \rule{\linewidth}{1pt}
       

   \end{center}
   
\end{titlepage}
\newpage
\section{Planteamiento general}
\textbf{1. Suponiendo que no hay obstáculos, probar que la conexión óptima entre las casas, así como la de la o las que correspondan con el enganche, será en línea recta y que no hay cruces entre los cables.} \\
Para empezar, vamos a suponer que el conjunto de casas y el punto de enganche conforman una red de puntos en el plano que llamaremos $S$. Al número de nodos de la red lo llamaremos $n$, y reservaremos $m$ para el número de aristas. \\
Para minimizar la longitud total de las aristas del grafo, podemos fijarnos en dos aspectos:
\begin{itemize}
    \item Reducir el número de aristas empleadas en el grafo al mínimo. 
    \item Utilizar las aristas con menor longitud. 
\end{itemize}


\end{document}





