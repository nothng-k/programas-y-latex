\documentclass[11pt, a4paper]{article} %tamaño mínimo de letra 11pto.
\usepackage{amssymb}
\usepackage{amsmath}
\usepackage{graphicx} 
\usepackage[spanish]{babel} %Español 
\usepackage[utf8]{inputenc} %Para poder poner tildes
\usepackage{vmargin} %Para modificar los márgene
\newcommand{\R}{\ensuremath{\mathbb{R}}}
\setmargins{2.5cm}{1.5cm}{16.5cm}{23.42cm}{10pt}{1cm}{0pt}{2cm}
%margen izquierdo, superior, anchura del texto, altura del texto, altura de los encabezados, espacio entre el texto y los encabezados, altura del pie de página, espacio entre el texto y el pie de página

\begin{document}
%%%%%%Portada%%%%%%%
\newpage

{\bfseries \large Capítulo 2 }\vspace{10mm} 

{\bfseries \large Base teórica} \vspace{15mm}

En éste capítulo se representan las nociones de teoría de juegos y análisis de sensibilidad, que aplicaremos posteriormente en el caso práctico 

Para la documentación en teoría de juego se han utilizado



{\bfseries \large 2.1. Teoría de juegos} \vspace{10mm}

En teoría de juegos cada jugador intenta conseguir el mejor resultado posible (maximizar su utilidad), pero teniendo en cuenta que el resultado del juego no depende sólo de sus acciones, sino también de las acciones de los otros jugadores. Para maximixar la utilidad supondremos que los jugadores son racionales, buscando un equilibrio en el que a los jugadores no les convenga cambiar su decisión .


El campo de estudio de la teoría de juegos es muy general. No es preciso que haya
entretenimiento, pero sí interacción.

{\bfseries \large 2.1.1. Teoría de la utilidad} \vspace{5mm}

empezaremos explicando las relaciones preferenciales(extenderse)

$\preceq 
\succeq$

las relaciones que vamos a describir se tratan de relaciones binarias entre diferentes alternativas

Definición 2.1 Relación preferencial: Sean $x$ e $y$ dos estados de un cierto espacio de estados, representaremos que x es al menos tan preferible como y si $x \succeq y$. 

Definición 2.2 Relación preferencial estricta: Sean $x$ e $y$ dos estados de un cierto espacio de estados, diremos que x es preferible a y, representándolo como $x \succ y$ si y solo si $x \succeq y$ pero no se da $x \preceq y$.

Definición 2.3 Relación indiferente: Sean $x$ e $y$ dos estados de un cierto espacio de estados, diremos que x es indiferente a y, representándolo como $x \sim y$ si y solo si se da  $x \succeq y$ y $x \preceq y$.

Definición 2.4 Preferencia racional:
Una relación de preferencia es racional si se cumplen las siguientes dos propiedades, dado el espacio de estados $X$:
(I) Completitud: Para todo $x,y \in X$ tenemos que $x \preceq y$ o $x \succeq y$ (o no excluyente)
(II)Transitividad: Para todo $x, y, z \in X$ tenemos que si $x \preceq y$ y $y \preceq z$ entonces $x \preceq z$


explicación de la función de utilidad

Definición 2.5 Función de utilidad: Una función $u:X \to \R$ es una función de utilidad que representa una relación si, para todo $x,y \in X$,

(Nótese que de  )


Proposición 2.6 
Una relación preferencial puede ser representada mediante una función de utilidad solo si es racional.

$x \preceq y \iff x \le y$ .

¿utilidades exponenciales?


{\bfseries \large 2.1.1. Teoría de la decisión} \vspace{5mm}

La teoría de la decisión analiza la forma de seleccionar una de las alternativas entre varia posibles de manera racional

definición 2.7 problema de decisión  
 Los elementos esenciales de un problema de decisión son:
  
  (I)Un conjunto, $A$, de acciones o alternativas entre las cuales el decisor debe elegir la que le parezca mejor.
  
  (II)Un conjunto, $\Theta$, de estados de la naturaleza , que describen las circunstancias que pueden afectar o influir en la decisiones a adoptar.
  
  (III)Una función de utilidad $u:Ax\Theta \to \R$ de modo que  para $\theta \in \Theta $ y $a \in A$, $u(\theta , a)$ mide las consecuencias de cada acción $a$ cuando el estado de la naturaleza es $\theta$.
   
De ahora en adelante supondremos que la función de utilidad proveniente de una relación preferencial.
 
Nótese que en (I) y en (II) los conjuntos pueden ser finitos, numerables, continuos o arbitrariamente complejos

Puede darse el caso de que el conjunto $\Theta$ tenga asociado una distribución de probabilidad(puede ser objetiva o subjetiva)

definición 2.8 Lotería simple: Dado un conjunto de estados de la naturaleza $\Theta =\{ \theta_1,...,\theta_N \} $ $N \in \mathbb{N}$ entonces se llama lotería a la lista $\Lambda =\{ \lambda_1,...,\lambda_N\} $ donde $\lambda_n \ge 0$ $\forall n \in \{ 1,..,N\} $ y $\sum_{n=1}^{N} \lambda_n=1$, donde $\lambda_n$ se interpreta como la probabilidad de que el suceso $\theta_n$ ocurra.



definición 2.9 Función de utilidad esperada Von Neumann-Morgenstern (VN-M):
En un problema de decisión se define utilidad esperada de una lotería $ \Lambda =\{ \lambda_1,...,\lambda_N\} $ dada una acción $a \in A$ dada  como:
$u_a(L)=\sum_{n=1}^{N}\lambda_n*u(a,n)$


{\bfseries \large 2.1. Teoría de juegos} \vspace{10mm}

Hasta ahora hemos analizado los problemas de decisión individuales, podemos interpretar como una progresión razonable del problema es añadir personas, es decir, además de una naturaleza aleatoria que puede alterar la decisión que más nos conviene, añadimos a otras personas que pueden alterarla también.

def 2.10 juego :
definimos juego como la representación formal de una situación donde un número de individuos, mayor que uno, interactuan de manera interdependiente.

los elementos básico de un problema en teoría de juego son:
(I)Los jugadores: Serán los que tomarán decisiones que modifiquen el juego (se puede considerar a la naturaleza aleatoria como uno).
(II)Las reglas: Definirá lo que puede y no puede hacer cada jugador y en que momento.
(III)Los resultados: Definirá cual es el resultado de cada acción de cada jugador.
(IV)Los beneficios: Define las preferencias de cada jugador sobre los posibles resultados, normalmente se usarán funciones de utilidad para definirlos.



Tipos de juegos

Hay dos enfoques básicos, juegos cooperativos y no cooperativos.En los juegos cooperativos se analizan las posibilidades de que algunos o todos los jugadores lleguen a un acuerdo sobre qué decisiones va a tomar cada uno, mientras que en el enfoque no cooperativo se analiza qué decisiones tomaría cada jugador en ausencia de acuerdo previo, nosotros nos centraremos en estos últimos.

También se pueden dividir los juego entre juego dinámicos y juegos estáticos

Hay principalmente dos formas de representar un juego:
(I)La normal (o estratégica) forma de representación donde la información es descrita mediante una tabla
(II)La forma extensiva de representación donde la información esta explicitamente descrita usando arboles de juego y conjuntos de información.

definición Game tree :
es un un grafo con estructura de árbol, consistente en los siguientes elmentos :

(I)Orden de movimientos:









\newpage

%%Inicio: 


\end{document}
