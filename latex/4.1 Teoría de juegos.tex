\documentclass[a4paper]{report} %tamaño mínimo de letra 11pto.
\usepackage{amssymb}
\usepackage{imakeidx}%indice
\usepackage{amsmath}
\usepackage{graphicx} 
\usepackage[T1]{fontenc}
%\usepackage[decisionutilitycolor]diagrama de influencia%

\makeindex[columns=3, title=Alphabetical Index, intoc]


\usepackage{hyperref}
\hypersetup{
    colorlinks=false,
    filecolor=magenta,      
    urlcolor=cyan,
    pdfpagemode=FullScreen,
    }

\usepackage[spanish]{babel} %Español 
\usepackage[utf8]{inputenc} %Para poder poner tildes
\usepackage{vmargin} %Para modificar los márgene
\newcommand{\R}{\ensuremath{\mathbb{R}}}


\newtheorem{definicion}{Definición}[section]

\newtheorem{propo}{Proposición}[section]


\begin{document}

%%%%%%Portada%%%%%%%
\tableofcontents

\listoffigures
\listoftables

\newpage

\chapter{Resumen}


\chapter{Introducción}

La toma de decisiones está muy presente en muchos de los aspectos de nuestra sociedad, cuando tenemos que decidir que producto comprar, una empresa decidiendo 
es frecuente que  nuestras decisiones afecten a los demás y viceversa, estudiar la mejor opción puede llegar a resultar complicado. Entre muchas complicaciones que pueden surgir , algunas son que los demás pueden moverse por intereses propios, por sentimientos o puede que incluso nos falten datos o que estos sean imprecisos. Cooperar puede ser una opción pero no siempre hay la confianza para hacerlo





Cuando se dan estas situaciones, estamos ante un conflicto de intereses que no tiene por qué tener fácil solución, dado que los demás podrían estar 








En este tfg estudiaré los problemas que pueden surgir en el caso de que no haya cooperación y tengamos información imperfecta, para tomar la decisión 

\chapter{Base teórica.}


En éste capítulo se representan, daremos una breve y necesaria introducción en teoría de la decisión y teoría de la utilidad, posteriormente daremos las nociones de teoría de juegos y análisis de sensibilidad que aplicaremos en el caso práctico.

Para la documentación en teoría de juego se han utilizado




\section{Teoría de la utilidad.}

empezaremos explicando las relaciones preferenciales(extenderse)


las relaciones que vamos a describir se tratan de relaciones binarias entre diferentes alternativas

\begin{definicion}[Relación preferencial]
 Sean $x$ e $y$ dos estados de un cierto espacio de estados, representaremos que x es al menos tan preferible como y mediante $x \succeq y$. 
\end{definicion}

\begin{definicion}[Relación preferencial estricta]
Sean $x$ e $y$ dos estados de un cierto espacio de estados, diremos que x es preferible a y, representándolo como $x \succ y$ si y solo si $x \succeq y$ pero no se da $x \preceq y$. 
\end{definicion}

\begin{definicion}[Relación indiferente]
Sean $x$ e $y$ dos estados de un cierto espacio de estados, diremos que x es indiferente a y, representándolo como $x \sim y$ si y solo si se da  $x \succeq y$ y $x \preceq y$.
\end{definicion}

\begin{definicion}[Preferencia racional]
Una relación de preferencia es racional si se cumplen las siguientes dos propiedades, dado el espacio de estados $X$:

(I) Completitud: Para todo $x,y \in X$ tenemos que $x \preceq y$ o $x \succeq y$ (o no excluyente)

(II)Transitividad: Para todo $x, y, z \in X$ tenemos que si $x \preceq y$ y $y \preceq z$ entonces $x \preceq z$.
\end{definicion}

\begin{definicion}[Función de utilidad]
Una función $u:X \to \R$ es una función de utilidad que representa una relación si, para todo $x,y \in X$:
$$ 
x \preceq y \iff u(x)\le u(y)
$$

\end{definicion}

\begin{propo}
Una relación preferencial puede ser representada mediante una función de utilidad solo si es racional.
\end{propo}









%¿utilidades exponenciales?

\section{Teoría de la decisión.}


La teoría de la decisión analiza la forma de seleccionar una de las alternativas entre varia posibles de manera racional
\begin{definicion}[Problema de decisión.]
Un problema de decisión, se encarga de analizar la forma de seleccionar una de las alternativas, en cierta situación, de forma racional.
Los elementos esenciales de un problema de decisión son:
  
  (I)Un conjunto, $A$, de acciones o alternativas entre las cuales el decisor debe elegir la que le parezca mejor.
  
  (II)Un conjunto, $\Theta$, de estados de la naturaleza , que describen las circunstancias que pueden afectar o influir en la decisiones a adoptar.
  
(III)Una función de utilidad $u:A \times  \Theta \to \R$ de modo que  para $\theta \in \Theta $ y $a \in A$, $u(\theta , a)$ mide las consecuencias de cada acción $a$ cuando el estado de la naturaleza es $\theta$.
\end{definicion}

   
De ahora en adelante supondremos que la función de utilidad proveniente de una relación preferencial.
 
Nótese que en (I) y en (II) los conjuntos pueden ser finitos, numerables, continuos o arbitrariamente complejos

Puede darse el caso de que el conjunto $\Theta$ tenga asociado una distribución de probabilidad(puede ser objetiva o subjetiva)
\begin{definicion}[Lotería simple]
Dado un conjunto de estados de la naturaleza $\Theta =\{ \theta_1,...,\theta_N \} $ $N \in \mathbb{N}$ entonces se llama lotería a la lista $\Lambda =\{ \lambda_1,...,\lambda_N\} $ donde $\lambda_n \ge 0$ $\forall n \in \{ 1,..,N\} $ y $\sum_{n=1}^{N} \lambda_n=1$, donde $\lambda_n$ se interpreta como la probabilidad de que el suceso $\theta_n$ ocurra.
\end{definicion}


\begin{definicion}[Función de utilidad esperada Von Neumann-Morgenstern)]
En un problema de decisión se define utilidad esperada de una lotería $ \Lambda =\{ \lambda_1,...,\lambda_N\} $ dada una acción $a \in A$ dada  como:
$u_a(L)=\sum_{n=1}^{N}\lambda_n*u(a,n)$

\end{definicion}

\section{Teoría de juegos.}


Hasta ahora hemos analizado los problemas de decisión individuales, podemos interpretar como una progresión razonable del problema es añadir personas, es decir, además de una naturaleza aleatoria que puede alterar la decisión que más nos conviene, añadimos a otras personas con sus propios intereses que pueden alterarla también.

En teoría de juegos cada jugador intenta conseguir el mejor resultado posible (maximizar su utilidad), pero teniendo en cuenta que el resultado del juego no depende sólo de sus acciones, sino también de las acciones de los otros jugadores. Para maximixar la utilidad supondremos que los jugadores son racionales, buscando un equilibrio en el que a los jugadores no les convenga cambiar su decisión .


Por último, no olvidemos que el campo de estudio de la teoría de juegos es muy general. No es preciso que haya entretenimiento, pero sí interacción.
\begin{definicion}[Juego]
Definimos juego como la representación formal de una situación donde un número de individuos, mayor que uno, interactuan de manera interdependiente.
Los elementos básico de un problema en teoría de juego son:

(I)Los jugadores: Serán los que tomarán decisiones que modifiquen el juego (se puede considerar a la naturaleza aleatoria como uno).

(II)Las reglas: Definirá lo que puede y no puede hacer cada jugador y en que momento.

(III)Los resultados: Definirá cual es el resultado de cada acción de cada jugador.

(IV)Los beneficios: Define las preferencias de cada jugador sobre los posibles resultados, normalmente se usarán funciones de utilidad para definirlos.

\end{definicion}


Llamaremos nodo a las situaciones de elección de alguno de los jugadores o el final del juego

\begin{definicion}[Conjunto de información de un jugador]
Un conjunto de información para un jugador es un conjunto de todas las acciones posibles que podrían realizar en una etapa determinada del juego dicho jugador, dado que el jugador no tiene por qué saber lo que han hecho los demás jugadores. Denotamos el conjunto de información que contiene el nodo de decisión $x$ por $H(x)$
\end{definicion}

 Es decir dependiendo de las acciones de los demás jugadores estaremos en una situación $x$ y podrá hacerse $H(x)$. Dado que un conjunto de información puede contener varios nodos de decisión, se deduce que si

\begin{definicion}[Estrategia]
Si denotamos el conjunto de información del jugador $i$ por $\mathcal{H_i}$, el conjunto de las posibles acciones por $\mathcal{A} $ y el conjunto de posibles acciones dada la información $H$ por $ C(H) \subset \mathcal{A} $. Entonces definimos estrategia de un jugador $i$ como la función $s_i:\mathcal{H}_i\longrightarrow \mathcal{A}$ con la propiedad de que $s_i(H) \in C(H)~~ \forall H \in \mathcal{H}_i $
\end{definicion}

%los jugadores son racionales porque patata%
\subsection{Clasificaciones de juegos.}

Hay dos enfoques básicos, juegos cooperativos y no cooperativos.En los juegos cooperativos se analizan las posibilidades de que algunos o todos los jugadores lleguen a un acuerdo sobre qué decisiones va a tomar cada uno, mientras que en el enfoque no cooperativo se analiza qué decisiones tomaría cada jugador en ausencia de acuerdo previo, nosotros nos centraremos en estos últimos.

Los juegos se pueden dividir los juego entre juego dinámicos y juegos estáticos, aunque se pueden entremezclar los juegos.

\begin{definicion}[Juego simultáneo]
Los juegos simultáneos son aquellos en los que los jugadores 
mueven a la vez o desconocen los movimientos anteriores de los 
otros jugadores. 
\end{definicion}


\begin{definicion}[Juego secuenciales]
Los juegos secuenciales los jugadores tienen algún 
conocimiento de las acciones previas. Los jugadores no tienen que 
tener una información perfecta, es suficiente con que tengan algo 
de información
\end{definicion}

Por ejemplo, el ajedrez se trata de un juego secuencia, ya un jugador mueve después del otro, mientras que el piedra, papel y tijeras es un juego simultáneo.


\begin{definicion}[Juego con información perfecta y con información imperfecta]
Los juegos con información imperfecta son los juegos secuenciales, en los que los jugadores no tienen por qué saber  con cada movimiento los movimientos que han realizado los demás jugadores con anterioridad, mientras que los juegos con información perfecta si.
\end{definicion}

Retomando el ejemplo del ajedrez, éste tendría información perfecta ya que en todo momento se sabe que ha realizado el otro jugador.

\begin{definicion}[Juego de suma cero]
Es aquel juego que al sumar todos los beneficios de todos los jugadores siempre de la misma constante.
\end{definicion}

Es decir, los recursos disponibles no se ven alterados por el juego, como en una partida de póker, el dinero en la mesa al final de la partida no cambia, sólo cambia de manos.

Hay muchas más formas de clasificación de juegos como los juegos bayesianos o los juegos de información completa, pero con esto es suficiente para desarrollar el trabajo.

\subsection{Modelos de representación.}

Hay diversas formas de representar un, las clásicas son dos, la forma normal y la forma extensiva (o de árbol), introduciremos otra forma de representación llamada diagrama de influencias:

\subsubsection{Forma normal.}


En un juego representado en forma normal, se muestran los jugadores, las estrategias, y las recompensas en una matriz. Un jugador elige las filas y el otro jugador elige las columnas. Cada jugador tiene diversas estrategias, que están especificadas por el número de filas y el número de columnas. Las recompensas que obtienen cada uno de ellos se muestran en el interior de la matriz. El primer número del par dado para cada una de las opciones es la recompensa recibida por el jugador de las filas y el segundo es la recompensa del jugador de las columnas. La representación de una matriz de resultados para un juego en su forma normal sería la siguiente: 

(insertar tabla)

Donde Ai y Bi representan los resultados para cada uno de los jugadores según la opción escogida por ambos. Por esto, la forma normal de representación se suele usar para representar los juegos simultáneos.

Si los jugadores tienen alguna información acerca de las elecciones de otros jugadores el juego se presenta habitualmente en la forma extensa. 

\subsubsection{Forma Extensiva.}

En la representación de juegos en forma extensa los juegos se presentan como árboles en los que cada vértice o nodo representa un punto donde el jugador toma decisiones. El jugador se especifica por un número situado junto al vértice. Las líneas que parten del vértice representan acciones posibles para el jugador. Las recompensas se especifican en las terminaciones de las ramas del árbol. 

Un juego en forma extendida se representa de la forma: 

(tabla)


En la representación anterior se muestra la forma extendida de un juego con dos jugadores. El jugador I mueve primero y puede elegir entre el movimiento A o B. El jugador II, según el movimiento del jugador I elige el movimiento C o D.

En función de las opciones elegidas por ambos jugadores se llega a los posibles resultados R1, R2, R3 o R4. En un juego representado en su forma extensa está definido el conjunto de reglas que fijan las posibles jugadas en todo momento, incluyendo qué jugador tiene que mover, la probabilidad de cada una de las opciones si las jugadas se hacen de forma aleatoria y el conjunto de resultados finales que relaciona una ganancia con cada una de las posibles formas de terminar el juego.

Además, se asume que cada jugador tiene unas preferencias para cada jugada de forma que obtenga la máxima ganancia (o las mínimas pérdidas). 

Los juegos representados de esta forma pueden modelar también juegos de movimientos simultáneos. En esos casos se dibuja una línea punteada o un círculo alrededor de dos vértices diferentes para representarlos como parte del mismo conjunto de información (por ejemplo, cuando los jugadores no saben en qué punto se encuentran). 

La representación en forma normal da al matemático una notación sencilla para el estudio de los problemas de equilibrio porque desestima la cuestión de cómo las estrategias son calculadas. Para tratar esto, es más conveniente usar la forma extensa del juego. 

\subsubsection{Diagramas de influencia.}





\section{Análisis de sensibilidad.}
Hay principalmente dos formas de representar un juego:
(I)La normal (o estratégica) forma de representación donde la información es descrita mediante una tabla
(II)La forma extensiva de representación donde la información esta explicitamente descrita usando arboles de juego y conjuntos de información.

\begin{definicion}[]

\end{definicion}






\newpage

%%Inicio: 


\end{document}
